% For LaTeX-Box: root = stat105_hw2.tex 
%%%%%%%%%%%%%%%%%%%%%%%%%%%%%%%%%%%%%%%%%%%%%%%%%%%%%%%%%%%%%%%%%%%%%%%%%%%%%%%%
%  File Name: stat105_hw2.tex
%  Purpose:
%
%  Creation Date: 03-09-2015
%  Last Modified: Wed Apr 13 11:06:08 2016
%  Created By:
%%%%%%%%%%%%%%%%%%%%%%%%%%%%%%%%%%%%%%%%%%%%%%%%%%%%%%%%%%%%%%%%%%%%%%%%%%%%%%%%

\documentclass[11pt]{article}
\usepackage{graphicx, fancyhdr}
\usepackage{amsmath, amsfonts}
\usepackage{color}

\newcommand{\blue}[1]{{\color{blue} #1}}

\setlength{\topmargin}{-.5 in} 
\setlength{\textheight}{9 in}
\setlength{\textwidth}{6.5 in} 
\setlength{\evensidemargin}{0 in}
\setlength{\oddsidemargin}{0 in} 
\setlength{\parindent}{0 in}
\newcommand{\ben}{\begin{enumerate}}
\newcommand{\een}{\end{enumerate}}


\lhead{STAT 105, Section A} 
\chead{Homework Assignment 7} 
\rhead{Due Tuesday, April 19} 
\lfoot{Spring 2016} 
\cfoot{\thepage} 
\rfoot{} 
\renewcommand{\headrulewidth}{0.4pt} 
\renewcommand{\footrulewidth}{0.4pt} 
\newcommand{\ans}[1]{{\color{blue} \textbf{Answer: } #1}}
\renewcommand{\ans}[1]{}

\def\Exp#1#2{\ensuremath{#1\times 10^{#2}}}
\def\Case#1#2#3#4{\left\{ \begin{tabular}{cc} #1 & #2 \phantom
{\Big|} \\ #3 & #4 \phantom{\Big|} \end{tabular} \right.}

\begin{document}
\pagestyle{fancy} 

Show \textbf{all} of your work on this assignment and answer each question fully in the given context. \\

\emph{Please} staple your assignment! \\

\ben


\item \textbf{Chapter 5, Section 4, Exercise 2 (page 300)}

\item \textbf{Chapter 5, Section 4, Exercise 3 (page 300)}

   Note: the table at the bottom left of page 300 has a typo: going down the $y$ column the values given are "1, 2, 3, 3" when they should be "1, 2, 3, 4."

   Hint: Understanding the process of testing for the toxic specimen is key.
   \begin{itemize}
      \item If there is NOT a contaminated speciman (the $X = 0$ case), then we will not find it on any test (cause it's not there).
      \item If there is a contaminated specimen (the $X = 1$ case):
         \begin{itemize}
         \item we may find it on the first test, the second test, the third test, or the fourth test.
         \item since there is no way to know which specimen is toxic, we can only randomly select them until we find the toxic one.
         \end{itemize}
   \end{itemize}

\item \textbf{Chapter 5 Exercise 35 (page 330)}

\een
\end{document}
