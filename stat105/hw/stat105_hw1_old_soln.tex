\documentclass[11pt]{article}
\usepackage{graphicx, fancyhdr}
\usepackage{amsmath, amsfonts}
\usepackage{color}

\newcommand{\blue}[1]{{\color{blue} #1}}

\setlength{\topmargin}{-.375 in} 
\setlength{\textheight}{8.75 in}
\setlength{\textwidth}{6.5 in} 
\setlength{\evensidemargin}{0 in}
\setlength{\oddsidemargin}{0 in} 
\setlength{\parindent}{0 in}
\newcommand{\ben}{\begin{enumerate}}
\newcommand{\een}{\end{enumerate}}

\lhead{Stat 105} 
\chead{Homework Assignment 1} 
\rhead{Due Thursday, January 19} 
\lfoot{Spring 2011} 
\cfoot{\thepage} 
\rfoot{} 
\renewcommand{\headrulewidth}{0.4pt} 
\renewcommand{\footrulewidth}{0.4pt} 

\def\Exp#1#2{\ensuremath{#1\times 10^{#2}}}
\def\Case#1#2#3#4{\left\{ \begin{tabular}{cc} #1 & #2 \phantom
{\Big|} \\ #3 & #4 \phantom{\Big|} \end{tabular} \right.}

\begin{document}
\pagestyle{fancy} 

Show \textbf{all} of your work on this assignment and answer each 
question
fully in the given context.

\emph{Please} staple your assignment!

%\vspace{1pc}
%For the first two problems, your choices for variable types are 
%response, supervised (controlled), supervised (experimental), and 
%lurking. 
%
%\blue{If we reword the questions, these instructions would be 
%eliminated.}

\ben

\item \textbf{Hockey game attendance.}

Caroline performs the following study to see if outside temperature
has an effect on attendance at her college's hockey games. For each
hockey game at her college, Caroline records the outside 
temperature and the attendance. Here are her results:

\hspace{1in}
\begin{tabular}{|cr@{/}r|c|c|} \hline
\multicolumn{3}{|c|}{\emph{Date}} & \emph{Temperature, deg.\ F} & 
\emph{Attendance} \\ \hline
Friday & 12&14 & 35 & 840 \\
Wednesday & 12&19 & 20 & 560 \\
Tuesday & 1&8 & $\!-5$ & 440 \\
Friday & 1&11 & 23 & 775 \\
Wednesday & 1&23 & 14 & 680 \\
Saturday & 2&2 & 30 & 950 \\
Friday & 2&8 & 28 & 950 \\
\hline
\end{tabular}
 
\ben 
\item Is this an experiment or an observational study?

\blue{
%\item What type of variable is attendance?
%
%\blue{
%I would prefer to reword this to be: Identify the response variable.
\item Identify the response variable.
%This would eliminate confusion with the terminology introduced in 
%chapter 1, e.g. categorical/quantitative variables. I know the 
%instructions seem clear to us, but I think this will make Fan's life 
%easier.
%}

%\item What type of variable is outside temperature?

%\blue{
%I agree that this is a factor, but it is not a supervised variable, 
%since Caroline does not exercise power over it; thus, we need to 
%rework this question. The problem here is trying to fit an 
%observational study into the framework of an experiment.
%}
\item What is the primary factor in this study?

\een

Caroline analyzes her results and finds that
outside temperature and attendance have a strong
positive correlation (i.e., as one increases, the other also 
increases).
She concludes that higher game day temperatures causes higher
attendance at their college's hockey games.

\ben
\addtocounter{enumii}{3}
\item Did she come to a proper conclusion for this study? Why or 
why not?

\item How does game attendance relate to the day of the week 
(regarding weekday games vs.\ weekend games)?

%\item Look at the day of the week of the hockey games.
%What type of variable is this?
\item What type of variable is day of the week of the hockey games?
 
%\item How does game attendance relate to the day of the week 
%(regarding weekday games vs.\ weekend games)?

\item For what type of studies do you have to worry
about possible lurking variables affecting the results?
\een

%\item \textbf{Extreme stretching.}
%  
%George works for a company that manufactures 
%rubber washers.  He randomly selects 100 
%washers off the assembly line for a study on the 
%durability of these washers under extreme 
%stretching.  To make sure that the washers 
%are fit to be used in the real world, George must test 
%the washers.  Holding heat constant, 
%George subjects each washer to one of various degrees of 
%stretching.  The washers are assigned to the different stretching
%groups randomly. After each test, George classifies 
%a washer as either defective or non-defective. 
% 
%\ben 
%\item Is this an experiment or observational study? 
% 
%\item What type of variable is heat? 
% 
%\item What type of variable is the amount of stretching? 
% 
%\item What type of data is recorded, quantitative or qualitative? 
% 
%\item The 100 selected washers constitutes the sample.
%What is the population? 
%\een
%
%George analyzes the results and finds no significant correlation
%between the defect rate and the amount of stretching.
%George concludes that the amount of stretching does
%not cause a change in the defect rate of the washers.
%
%\ben
%\addtocounter{enumii}{5}
%\item Did he come to a proper conclusion for this study? Why or 
%why not?
%\een

\item \textbf{Mathematics curricula.}

Three high schools in Iowa participated in a study to evaluate the 
effectiveness of a new computer-based mathematics curriculum. In 
each school, four 24-student sections of freshman algebra were 
available for study. The two types of instruction (standard 
curriculum, computer-based curriculum) were randomly assigned to 
the four sections in each of the three schools, resulting in each 
instruction method being assigned to two sections within each
school. 
%\blue{Cory: 2 of each treatment per school,correct? Perhaps make
%this point clear?}
At the end of the 
term, a standard mathematics achievement test was given to each of 
the 24 students in each section.

\ben
\item Is this an experiment or an observational study? Explain 
briefly.
\item Identify the experimental units.
\item Identify all experimental variables, settings of the 
experimental variables, and treatments.
\item What type of experimental design is being implemented here?
\item The twelve sections of freshman algebra constitutes the sample. 
What is the population?

\een

\item \textbf{Microwave popcorn.}

A study is to be carried out to determine the optimal combination of 
microwave oven settings for microwave popcorn. Cooking time has three 
possible settings (3, 4, and 5 minutes) and cooking power has two 
settings (low power, high power). The response (to be minimized) is 
the number of burned plus the number of unpopped kernels. 18 bags of 
popcorn are available for this experiment.

\ben
\item Is this an experiment or an observational study? 
Explain briefly.
\item Identify the experimental units.
\item  Identify the experimental variable(s), settings of the experimental variables, and treatments.
\item Identify two potential blocking factors.
\item Describe how randomization would be performed in this study if 
3 bags of popcorn are assigned to each treatment group.
\item Using table B.1 (starting in row 6), carry out the randomization 
process you described above. To present your results tabulate the 
treatment and the bag IDs assigned to that treatment.
\een


\item \textbf{Measurement characteristics.}

Students Jack and Jill each use the same ruler to measure
the thicknesses of various books 
book to the nearest millimeter.  All books have the same thickness 
of 3.6 cm,
a fact that is unknown to them.
After doing the measurement five times, 
the following data are recorded: 
\begin{tabbing}
\qquad \=Jack: \ \ \= 3.5 cm, 3.7 cm, 4.0 cm, 3.3 cm, 3.5 cm \\
 \> Jill:\> 3.9 cm, 3.8 cm, 4.0 cm, 3.9 cm, 3.8 cm 
\end{tabbing}
 
\ben
\item Who has more accurate data? 
 
\item Who has more precise data? 
\een

\een
\end{document}
