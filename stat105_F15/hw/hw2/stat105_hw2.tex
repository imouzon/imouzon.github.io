% For LaTeX-Box: root = stat105_hw2.tex 
%%%%%%%%%%%%%%%%%%%%%%%%%%%%%%%%%%%%%%%%%%%%%%%%%%%%%%%%%%%%%%%%%%%%%%%%%%%%%%%%
%  File Name: stat105_hw2.tex
%  Purpose:
%
%  Creation Date: 03-09-2015
%  Last Modified: Fri Jan 22 11:21:23 2016
%  Created By:
%%%%%%%%%%%%%%%%%%%%%%%%%%%%%%%%%%%%%%%%%%%%%%%%%%%%%%%%%%%%%%%%%%%%%%%%%%%%%%%%

\documentclass[11pt]{article}
\usepackage{graphicx, fancyhdr}
\usepackage{amsmath, amsfonts}
\usepackage{color}

\newcommand{\blue}[1]{{\color{blue} #1}}

\setlength{\topmargin}{-.5 in} 
\setlength{\textheight}{9 in}
\setlength{\textwidth}{6.5 in} 
\setlength{\evensidemargin}{0 in}
\setlength{\oddsidemargin}{0 in} 
\setlength{\parindent}{0 in}
\newcommand{\ben}{\begin{enumerate}}
\newcommand{\een}{\end{enumerate}}


\lhead{STAT 105, Section B} 
\chead{Homework Assignment 2} 
\rhead{Due Thursday, September 10} 
\lfoot{Fall 2015} 
\cfoot{\thepage} 
\rfoot{} 
\renewcommand{\headrulewidth}{0.4pt} 
\renewcommand{\footrulewidth}{0.4pt} 
\newcommand{\ans}[1]{{\color{blue} \textbf{Answer: } #1}}
\renewcommand{\ans}[1]{}

\def\Exp#1#2{\ensuremath{#1\times 10^{#2}}}
\def\Case#1#2#3#4{\left\{ \begin{tabular}{cc} #1 & #2 \phantom
{\Big|} \\ #3 & #4 \phantom{\Big|} \end{tabular} \right.}

\begin{document}
\pagestyle{fancy} 

Show \textbf{all} of your work on this assignment and answer each question fully in the given context. \\

\emph{Please} staple your assignment! \\

You will want to understand Exercise 1 from Section 2.1 before attempting the following questions.  
Your answer should be written in complete sentences.
It is possible that a drawing or table may help make your thoughts more concrete or illustrate a concept that would be difficult to describe in words alone - if so I encourage you to use one.

\ben

\item \textbf{Chapter 2, Section 3, Exercise 1 (page 47)}
\item \textbf{Chapter 2, Section 3, Exercise 5 (page 47)}
\item \textbf{Chapter 2, Section 4, Exercise 2 (page 56)}
\item \textbf{Chapter 2, Exercise 7 (page 65)}
\item \textbf{Chapter 2, Exercise 11 (page 65)}



\clearpage

\item \textbf{JMP Assignment.} 

   Without laboring the point, computing is one of the most important parts of modern data analysis. A large part of data science simply wouldn't exist without the tools developed by scientists working at the intersections of computer science, mathematics, and statistics. 
   Because of that, there will inevitably be parts of this course where a statistical computing tools are needed. SAS and R are the two main languages used by statisticians, with Python, Julia, F\#, C++ and others making important contributions as well.
   SAS has a software called JMP ("Jump") that makes doing statistical analyses simpler - it is more powerful than Excel or your calculator but requires little in the sense of coding making the learning curve much lower. 
   We will be using it this semester. There are labs in Snedecor Hall with the software pre-installed, but it is free for students and I encourage you to download a copy for yourself using the link below.

   Download: \href{http://www.stat.iastate.edu/resources-2/software-sasjmpr/statistical-software-jmp/}{http://www.stat.iastate.edu/resources-2/software-sasjmpr/statistical-software-jmp/}

   Additionally, you may want to consider the following tutorials (they are very helpful):
   
   Tutorials: \href{http://web.utk.edu/~cwiek/201Tutorials/}{http://web.utk.edu/~cwiek/201Tutorials/} 

   The tutorials cover the following topics:
   \begin{itemize}
     \item Histogram and Box Plot
     \item Stem and Leaf Plot
     \item Normal Probability Plot and Goodness of Fit Test
     \item Calculating Summary Statistics of Quantitative Data
     \item Getting JMP Graphics into Microsoft Word
   \end{itemize}

   For this problem I am asking you to:

   \begin{enumerate}
      \item Download and install \texttt{JMP} or find a computer with it already installed.
      \item Take a screen shot once you have it open. Print the screen shot and attach it to your homework.
   \end{enumerate}

\een
\een
\end{document}
