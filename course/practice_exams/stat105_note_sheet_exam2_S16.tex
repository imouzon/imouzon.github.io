\documentclass[10pt,landscape]{article}
\usepackage{multicol}
\usepackage{amsmath}
\usepackage{calc}
\usepackage{ifthen}
\usepackage[landscape]{geometry}
\usepackage{hyperref}

% To make this come out properly in landscape mode, do one of the following
% 1.
%  pdflatex latexsheet.tex
%
% 2.
%  latex latexsheet.tex
%  dvips -P pdf  -t landscape latexsheet.dvi
%  ps2pdf latexsheet.ps


% If you're reading this, be prepared for confusion.  Making this was
% a learning experience for me, and it shows.  Much of the placement
% was hacked in; if you make it better, let me know...


% 2008-04
% Changed page margin code to use the geometry package. Also added code for
% conditional page margins, depending on paper size. Thanks to Uwe Ziegenhagen
% for the suggestions.

% 2006-08
% Made changes based on suggestions from Gene Cooperman. <gene at ccs.neu.edu>


% To Do:
% \listoffigures \listoftables
% \setcounter{secnumdepth}{0}


% This sets page margins to .5 inch if using letter paper, and to 1cm
% if using A4 paper. (This probably isn't strictly necessary.)
% If using another size paper, use default 1cm margins.
\ifthenelse{\lengthtest { \paperwidth = 11in}}
	{ \geometry{top=.5in,left=.5in,right=.5in,bottom=.5in} }
	{\ifthenelse{ \lengthtest{ \paperwidth = 297mm}}
		{\geometry{top=1cm,left=1cm,right=1cm,bottom=1cm} }
		{\geometry{top=1cm,left=1cm,right=1cm,bottom=1cm} }
	}

% Turn off header and footer
\pagestyle{empty}
 

% Redefine section commands to use less space
\makeatletter
\renewcommand{\section}{\@startsection{section}{1}{0mm}%
                                {-1ex plus -.5ex minus -.2ex}%
                                {0.5ex plus .2ex}%x
                                {\normalfont\large\bfseries}}
\renewcommand{\subsection}{\@startsection{subsection}{2}{0mm}%
                                {-1explus -.5ex minus -.2ex}%
                                {0.5ex plus .2ex}%
                                {\normalfont\normalsize\bfseries}}
\renewcommand{\subsubsection}{\@startsection{subsubsection}{3}{0mm}%
                                {-1ex plus -.5ex minus -.2ex}%
                                {1ex plus .2ex}%
                                {\normalfont\small\bfseries}}
\makeatother

% Define BibTeX command
\def\BibTeX{{\rm B\kern-.05em{\sc i\kern-.025em b}\kern-.08em
    T\kern-.1667em\lower.7ex\hbox{E}\kern-.125emX}}

% Don't print section numbers
\setcounter{secnumdepth}{0}


\setlength{\parindent}{0pt}
\setlength{\parskip}{0pt plus 0.5ex}

% -----------------------------------------------------------------------

\begin{document}

\raggedright
\footnotesize
\begin{multicols}{2}


% multicol parameters
% These lengths are set only within the two main columns
%\setlength{\columnseprule}{0.25pt}
\setlength{\premulticols}{1pt}
\setlength{\postmulticols}{1pt}
\setlength{\multicolsep}{1pt}
\setlength{\columnsep}{2pt}

\begin{center}
   \Large{\textbf{STAT 105 Exam II}} \\
   \Large{\textbf{Reference Sheet}} \\
\end{center}

\section{Factorial Analysis (Two Factors)}

Assuming 

\begin{itemize}
   \item Factor A with levels $1, 2, ..., I$, 
   \item Factor B with levels $1, 2, ..., J$, 
   \item $n$ is the total number of observations,
   \item $n_{ij}$ is the total number of observations with Factor A at level $i$ and Factor B at level $j$, 
   \item $n_{i \cdot }$ is the total number of observations with Factor A at level $i$,
   \item $n_{\cdot j}$ is the total number of observations with Factor B at level $j$.
   \item $y_{ijk}$ is the $k$th observation where Factor A is at level $i$ and Factor B is at level $j$.
\end{itemize}

\begin{center}
\begin{tabular}{@{}ll@{}}
        & \\
   $y_{i j \cdot} = \sum_{k = 1}^{n_{ij}} y_{ijk} $ & $\bar{y}_{i j} = \frac{1}{n_{ij}} \sum_{k = 1}^{n_{ij}} y_{ijk} $ \\
        & \\
   $\bar{y}_{i \cdot} = \frac{1}{J} \sum_{j = 1}^J \bar{y}_{ij} $ & $\bar{y}_{\cdot j} = \frac{1}{I} \sum_{i = 1}^I \bar{y}_{ij} $ \\
        & \\
\end{tabular}
\end{center}

   $$\bar{y}_{\cdot \cdot} = \frac{1}{I} \sum_{i = 1}^I \bar{y}_{i \cdot} = \frac{1}{J} \sum_{j = 1}^J \bar{y}_{\cdot j} $$ \\

\begin{center}
\begin{tabular}{@{}ll@{}}
   Main effect of Factor A at level $i$ & $a_i = \bar{y}_{i \cdot} - \bar{y}_{\cdot \cdot}$ \\
        & \\
   Main effect of Factor B at level $j$ & $b_j = \bar{y}_{\cdot j} - \bar{y}_{\cdot \cdot}$ \\
        & \\
   Interaction of Factor B at level $j$  and Factor A at level $i$ & $ab_{ij} = \bar{y}_{ij} - a_i - b_j + \bar{y}_{\cdot \cdot}$ \\
        & \\
   Fitted Value (no interactions) & $\hat{y}_{ij} = a_i + b_j + \bar{y}_{\cdot \cdot}$ 
        & \\
   Fitted Value (including interactions) & $\hat{y}_{ij} = a_i + b_j + ab_{ij} + \bar{y}_{\cdot \cdot}$
        & \\
\end{tabular}
\end{center}

% \section{Factorial Analysis (Two Factors)}
% 
% \begin{itemize}
%    \item Factor A with levels $1, 2, ..., I$, 
%    \item Factor B with levels $1, 2, ..., J$, 
%    \item Factor C with levels $1, 2, ..., K$, 
%    \item $n$ is the total number of observations,
%    \item $n_{ijk}$ is the total number of observations with Factor A at level $i$ and Factor B at level $j$, 
%    \item $n_{ij\cdot}$ is the total number of observations with Factor A at level $i$ and Factor B at level $j$, 
%    \item $n_{i\cdot k}$ is the total number of observations with Factor A at level $i$ and Factor C at level $k$, 
%    \item $n_{\cdot jk}$ is the total number of observations with Factor B at level $j$ and Factor C at level $k$, 
%    \item $n_{i \cdot \cdot}$ is the total number of observations with Factor A at level $i$,
%    \item $n_{\cdot j \cdot}$ is the total number of observations with Factor B at level $j$,
%    \item $n_{\cdot \cdot k}$ is the total number of observations with Factor C at level $k$,
%    \item $y_{ijkl}$ is the $l$th observation where Factor A is at level $i$, Factor B is at level $j$, and Factor C is at level $k$.
% \end{itemize}
% 
% \begin{tabular}{@{}ll@{}}
%         & \\
%    $y_{\cdot \cdot \cdot} = \sum_{i = 1}^I \sum_{j = 1}^J \sum_{k=1}^K \sum_{l=1}^L y_{ijkl} $ & $\bar{y}_{\cdot \cdot \cdot} = \frac{1}{n} y_{\cdot \cdot \cdot} $ \\
%         & \\
%    $\bar{y}_{i j \cdot} = \frac{1}{n_{ij\cdot}} \sum_{k = 1}^K \sum_{l=1}^L y_{ijkl} $ & $\bar{y}_{i \cdot k} = \frac{1}{n_{i \cdot k}} \sum_{j = 1}^J \sum_{k=1}^K y_{ijk} $ \\
%         & \\
%    $\bar{y}_{\cdot j k} = \frac{1}{n_{\cdot jk}} \sum_{i = 1}^I \sum_{l=1}^L y_{ijkl} $ & $\bar{y}_{i \cdot \cdot} = \frac{1}{n_{i \cdot \cdot}} \sum_{j = 1}^J \sum_{k=1}^K y_{ijk} $ \\
%         & \\
%    Main effect of Factor A at level $i$ & $a_i = \bar{y}_{i \cdot} - \bar{y}_{\cdot \cdot}$ \\
%         & \\
%    Main effect of Factor B at level $j$ & $b_j = \bar{y}_{\cdot j} - \bar{y}_{\cdot \cdot}$ \\
%         & \\
%    Fitted Value & $\hat{y}_{ij} = a_i + b_j + \bar{y}_{\cdot \cdot}$
%         & \\
% \end{tabular}

\subsection{Basic Probability Rules}

\begin{tabular}{@{}ll@{}}
        & \\
   Probability $A$ given $B$ & $P[A | B] = \frac{P[A, B]}{P[B]}$ \\
        & \\
   Probability $A$ and $B$ & $P[A, B] = P[A | B] P[B] = P[B | A] P[A]$ \\
        & \\
   Probability $A$ or $B$ & $P[A \text{ or } B] = P[A] + P[B] - P[A, B]$ \\
        & \\
\end{tabular}

\vspace{.2cm}

\section{Discrete Random Variables}

\begin{tabular}{@{}ll@{}}
        & \\
   Probability function &  $P[X = x] = f_X(x)$ \\
        & \\
   Cumulative probability function &  $P[X \le x] = F_X(x)$ \\
        & \\
   Expected Value & $\mu = E(X) = \sum_{x} x f_X(x)$ \\
        & \\
   Variance & $\sigma^2 = Var(X) = \sum_{x} (x - \mu)^2 f_X(x)$ \\
        & \\
   Standard Deviation & $\sigma = \sqrt{Var(X)}$ \\
        & \\
\end{tabular}

\vspace{.2cm}

\subsection{Geometric Random Variables}

$X$ is the trial count upon which the first successful outcome is observed performing independent trials with probability of success $p$.

\begin{tabular}{@{}ll@{}}
        & \\
   Possible Values & $x = 1, 2, 3, \ldots$ \\
        & \\
   Probability function &  $P[X = x] = f_X(x) = p^x (1-p)^{x-1}$ \\
        & \\
   Expected Value & $\mu = E(X) = \frac{1}{p} $ \\
        & \\
   Variance & $\sigma^2 = Var(X) = \frac{1 - p}{p^2}$ \\
        & \\
\end{tabular}

\vspace{.2cm}

\subsection{Binomial Random Variables}

$X$ is the number of successful outcomes observed in $n$ independent trials with probability of success $p$.

\begin{tabular}{@{}ll@{}}
        & \\
   Possible Values & $x = 0, 1, 2, \ldots, n$ \\
        & \\
   Probability function &  $P[X = x] = f_X(x) = \frac{n!}{(n-x)!x!} p^x (1-p)^{n-x}$ \\
        & \\
   Expected Value & $\mu = E(X) = n p $ \\
        & \\
   Variance & $\sigma^2 = Var(X) = n p (1-p)$ \\
        & \\
\end{tabular}

\vspace{.2cm}

\subsection{Poisson Random Variables}

$X$ is the number of times a rare event occurs over a predetermined interval (an area, an amount of time, etc.) where the number of events we expect is $\lambda$.

\begin{tabular}{@{}ll@{}}
        & \\
   Possible Values & $x = 0, 1, 2, 3, \ldots$ \\
        & \\
   Probability function &  $P[X = x] = f_X(x) = \frac{e^{-\lambda} \lambda^x}{x!}$ \\
        & \\
   Expected Value & $E(X) = \lambda $ \\
        & \\
   Variance & $Var(X) = \lambda$ \\
        & \\
\end{tabular}
\vspace{.2cm}

\section{Continuous Random Variables}

\begin{tabular}{@{}ll@{}}
        & \\
   Probability density function &  $P[a \le X \le b] = \int_a^b f_X(x) dx$ \\
        & \\
   Cumulative probability function &  $P[X \le x] = F_X(x) = \int_{-\infty}^x f_X(t)dt$ \\
        & \\
   Expected Value & $\mu = E(X) = \int_{-\infty}^{\infty} x f_X(x) dx$ \\
        & \\
   Variance & $\sigma^2 = Var(X) = \int_{-\infty}^{\infty} (x - \mu)^2 f_X(x) dx$ \\
        & \\
   Standard Deviation & $\sigma = \sqrt{Var(X)}$ \\
        & \\
\end{tabular}

\vspace{.2cm}

\end{multicols}
\end{document}
