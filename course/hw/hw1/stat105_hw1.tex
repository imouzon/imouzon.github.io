% For LaTeX-Box: root = stat105_hw1.tex 
%%%%%%%%%%%%%%%%%%%%%%%%%%%%%%%%%%%%%%%%%%%%%%%%%%%%%%%%%%%%%%%%%%%%%%%%%%%%%%%%
%  File Name: stat105_hw1.tex
%  Purpose:
%
%  Creation Date: 14-01-2016
%  Last Modified: Thu Jan 28 12:04:13 2016
%  Created By:
%%%%%%%%%%%%%%%%%%%%%%%%%%%%%%%%%%%%%%%%%%%%%%%%%%%%%%%%%%%%%%%%%%%%%%%%%%%%%%%%

\documentclass[11pt]{article}
\usepackage{graphicx, fancyhdr}
\usepackage{amsmath, amsfonts}
\usepackage{color}

\newcommand{\blue}[1]{{\color{blue} #1}}

\setlength{\topmargin}{-.5 in} 
\setlength{\textheight}{9 in}
\setlength{\textwidth}{6.5 in} 
\setlength{\evensidemargin}{0 in}
\setlength{\oddsidemargin}{0 in} 
\setlength{\parindent}{0 in}
\newcommand{\ben}{\begin{enumerate}}
\newcommand{\een}{\end{enumerate}}


\lhead{STAT 105, Section A} 
\chead{Homework Assignment 1} 
\rhead{Due Friday, January 22 at 5:00 pm} 
\lfoot{Spring 2016} 
\cfoot{\thepage} 
\rfoot{} 
\renewcommand{\headrulewidth}{0.4pt} 
\renewcommand{\footrulewidth}{0.4pt} 
\newcommand{\ans}[1]{{\color{blue} \textbf{Answer: } #1 (5 pts.)}}
\renewcommand{\ans}[1]{}

\def\Exp#1#2{\ensuremath{#1\times 10^{#2}}}
\def\Case#1#2#3#4{\left\{ \begin{tabular}{cc} #1 & #2 \phantom
{\Big|} \\ #3 & #4 \phantom{\Big|} \end{tabular} \right.}

\begin{document}
\pagestyle{fancy} 

Show \textbf{all} of your work on this assignment and answer each 
question
fully in the given context.

\emph{Please} staple your assignment!

\ben

\item \textbf{Chapter 1, Exercise 1 (page 23)} \\

\ans{
      (5 pts.) During calibration, our goal is to get the equipment to reflect the truth. 
      This involves the use of standards that have some known quantity that should be measuring.
      Our main goal in this process is to be \textit{accurate}, meaning that, as the book states on page 17, our instrument is to bring our measurements in line with the standard.
}

\item \textbf{Chapter 1, Exercise 9 (page 24)}

\ans{In this case, we can take the use the scare-quotes around the word "the" as a possible hint - while there may be some \textbf{ideal version} of the dowel in question, we have real, incosistent, physical dowels. 
   The exact nature of each dowel, though identical in the large picture, may be less than identical in the details - slight differences in the texture or moisture content of the wood for instance may not have a great influence on the dowels strength, but may be just enough to make the strengths of any two dowels non-identical.
   Measuring the strengths of several of these dowels may help us get a better picture of what "the" strength of the ideal dowel we have in mind actually is.

   Of course, even if each dowel were exactly the same, we have the issue of measurement error. Since we hope that the machine is well calibrated, we can imagine that there is no systematic flaw in the way the observed measurements differ from the true quantities we are measuring.
   We can write this idea in an equation:

   $$ \text{Observed Measurement} = \text{Acutal Quantity} + \text{Random Measurement Error} $$

   If the error is truly random, making multiple measurements allows us to get a better idea what part of the measurements comes from the truth and what comes from the error.

   In both cases, testing several dowels gives us a chance to determine what the true value is by account for the existence of errors.
}


\item \textbf{Lurking variables.}

   There is an common saying in the sciences about establishing cause and effect: "correlation does not imply causation". 
   It summarizes the idea that two events can have a relationship without one causing the other.
   Examples are easy to think of - suppose I did an observational study in which I looked at the percent of people using an umbrella and the percent of people using windshield wipers, recording the two percentages every hour of the day for two weeks.
   After collecting my data, I will discover that every time the percent of people using umbrellas rose the percent of people using windshield wipers also rose. 
   The reason for this is not that people using umbrellas cause people to use windshield wipers.
   Instead both umbrella use and windshield wiper use are responses to a third variable, specifically whether or not it is raining.
   The danger in a scenario like this is that by just recording the percentage of people using umbrellas and the percentage of people using windshield wipers, I may falsely determine that one causes the other.
   In statistics we call variables that are unobserved but potentially the true causative factor \textbf{lurking variables} - variables that are not directly seen in the data set, but whose influence is.

   Consider the following cases where two variables are gathered and a pattern in the data is noticed.
   A cause and effect relationship is suggested between the two variables.
   Suggest a possible lurking variable in each case that might explain why the cause and effect relationship suggested could be wrong.

   \textbf{note}: this question is fairly open ended - as long as you attempt to explain how the lurking variable you suggest could be driving both of the other variables you will get full credit. For example, in the umbrella/windshield wiper example, a satisfactory response would be ``The presence of rain may be a lurking variable. When it rains, people use umbrellas to stay dry and windshield wipers in order to drive safely. When it is not raining, neither umbrellas or windshield wipers are needed. Because of this, both percentages would be higher when it is raining and lower when it is not raining."

   \ben
   \item 
   Researchers collecting data on outdoor temperature and the number of cases of common illnesses notice that when the temperature is lower more people get sick and when the temperature is higher fewer people get sick.
   They suggest that cold temperatures are causing these illnesses.

   \ans{ During the winter, people also tend to spend more time inside. Perhaps it is the \textbf{amount of time spent inside}, in closer contact with other people who may be sick which causes the change in the number of people falling ill.}

   \item 
   Researchers collecting data on work history notice that people are more likely to pass away in the five years after retirement than they are to pass away during any five year window of their careers.
   The conclude that retirement causes the person to pass away.
   \ans{People who retire are also generally older. If a person has no plans to retire any time soon, they may be earlier in their career. So, perhaps \textbf{age} is a potential lurking variable.}
 
   \item 
   Over five years, the Coast Guard gathered monthly data for the number of swimsuit sales and the number of shark attacks. They noticed that when the number of swimsuits sold in a month was higher, the number of shark attacks in that month was also higher.
   They concluded that swimsuit sales were causing shark attacks.
   \ans{In this case the a potential lurking variable is the \textbf{the number of people swimming in the ocean} or perhaps, the season. In the summer people not only get themselves new swimsuits but they also get in the ocean with sharks in those swimsuits.}

   \een



\item \textbf{Hockey game attendance.}

Caroline performs the following study to see if outside temperature
has an effect on attendance at her college's hockey games. For each
hockey game at her college, Caroline records the outside temperature and the attendance. Here are her results:

\hspace{1in}
\begin{tabular}{|cr@{/}r|c|c|} \hline
\multicolumn{3}{|c|}{\emph{Date}} & \emph{Temperature, deg. F} & \emph{Attendance} \\ \hline
Friday & 12&14 & 35 & 840 \\
Wednesday & 12&19 & 20 & 560 \\
Tuesday & 1&8 & $\!-5$ & 340 \\
Friday & 1&11 & 23 & 775 \\
Wednesday & 1&23 & 14 & 680 \\
Saturday & 2&2 & 30 & 950 \\
Friday & 2&8 & 28 & 950 \\
\hline
\end{tabular}
 
\ben 
\item Is this an experiment or observational study?
\ans{Observational study. Caroline's role is passive; 
there is no assignment of temperatures to the games.
In an experiment, the temperatures would be manipulated
to make a greater temperature difference between the games,
making the results more telling of the underlying relationship
between attendance and outside temperature.}

\item What type of variable is attendance?
\ans{The response (this is the focus of the study).}

\een

Caroline analyzes her results and finds that
outside temperature and attendance have a strong
positive correlation (i.e., as one increases, the other also increases).
She concludes that higher game day temperatures causes higher
attendance at their college's hockey games.

\ben
\addtocounter{enumii}{2}
\item Did she come to a proper conclusion for this study? Why or why not?
\ans{No, you cannot come up with a causation conclusion for observational studies%
---you can do so only for experiments. There may be (and is) a lurking variable responsible
for the effect on attendance that is also correlated with outside temperature,
thus making it \emph{seem} that the temperature is causing the change in
the response.}

\item Look at the day of the week of the hockey games.
What type of variable is this?
\ans{This is a lurking variable as it could affect the response (attendance). }
 
\item 
Rewrite the data table, adding a new column ``School Night" (using the values ``no" if the game is on a Friday or Saturday, and ``yes" if the game is on any other day).
How does Attendance relate to School Night?
\ans{Attendance is noticeably higher for weekend games, during which
the weather is also warmer. It may be that games have higher attendance
when they are played on the weekend or when the weather is warmer.
We do not know from this given study.}

\item For what type of studies do you have to worry
about possible lurking variables affecting the results?
\ans{Observational studies. 
%due to the lack of randomization of the
%units to the study treatments. (Here temperature was actually
%randomized properly---temperature is not truly
%correlated with the day of the week, though that would be nice.)
These results demonstrate the problem of drawing conclusions from
observational studies.}
\een

\item \textbf{Washer stretching.}
  
George works for a company that manufactures 
rubber washers.  He randomly selects 1000 
washers off the assembly line throughout two weeks
for a study on the durability of these washers under  
stretching.  To make sure that the washers 
are fit to be used in the real world, George must test 
the washers.  Holding heat constant, 
George subjects each washer to one of various methods of 
stretching.  The washers are randomly assigned to be stretched under one of five different forces (low, medium-low, medium, medium-high, and high).
After each test, George classifies 
a washer as either defective or non-defective. 
 
\ben 
\item Is this an experiment or observational study? 
\ans{Experiment. George manipulates the washers by assigning them
to certain treatment groups. In an observational study, one does not assign
units to treatment groups.}

\item What type of variable is heat? 
\ans{This is a controlled variable as heat is kept constant for all units. Controlled
variables are a type of supervised variables.}
 
\item What type of variable is the amount of stretching? 
\ans{The amount of stretching is the factor under study
(the variable George wants to see if it affects the response), 
thus it is an experimental variable (which is a type of supervised variable).}
 
\item What type of variable is response to the stretching method?
\ans{Qualitative data is recorded as the possible results (defective or non-defective)
are categories, not numbers.}
 
\item
The 100 selected washers constitutes the sample. What is the population? 
\ans{The population is all the rubber washers produced by George's company,
specifically all washers produces by this assembly line during these two weeks.
The company doesn't really care how these specific 100 washers perform--%
they want to learn about how their washers perform in general to stretching.} 

\item George analyzes the results and finds that the defect rate
increases with the amount of stretching.
Can George conclude that the amount of stretching 
causes a change in the defect rate of the washers?
Why or why not?
\ans{George can make a causation statement here.
As long as he ensures that the extraneous variables are properly
handled, either through controlling or adequate randomization of the units
and testing order, then he has a high degree of confidence that
stretching of the washers does affect the defect rate.}

%Remember, though, that there always exists the chance that the sample
%was not properly representative of the population, and perhaps 
%the 100 washers he selected 
%were some of the few washers in the population
%that showed no effect. 
%But through proper
%experimental design, this chance is minimized and George can 
%be highly confident in the results. More on confidence in Chapter 6.
\een

%\item
%\textbf{A Medical Study}
%
%A medical researcher wants to know how humans 
%will respond to a new growth hormone. 
%As human experimentation is difficult, 
%she instead proposes an experimental study 
%on rats to explore potential growth effects. 
%This researcher suspects that adequate protein intake 
%may be essential for the new growth hormone to be effective.  
%24 rats are split into two groups of 12 rats at random. 
%Each treatment group is given either 1g of protein per day 
%or 5g of protein per day.  All rats are injected daily 
%with the same dose of growth hormone. 
%The number of calories per day are set constant for each rat. 
%At the end of 6 weeks, she weighs each rat and
%compares the increase in weight of the 5g protein group compared with 
%that of the 1g protein group.
%A greater difference in the weight gain
%would indicate the importance of proper protein diet. 
%
%Fill out the information below about this experiment:
%\ben
%\item
%Response: (What is the outcome?)
%\ans{Weight gain.}
%
%\item
%Factors: (What is manipulated or imposed?)  
%\ans{Protein level in diet.}
%
%\item
%List the levels of each factor:
%\ans{1g/day, 5g/day.}
% 
%\item
%How many treatments are there?
%\ans{Two treatments: 5g/day group, 1g/day group.}
%
%\item
%How many Experimental Units? 
%\ans{There are 24 rats that are experimental units.}
%
%\item
%List 5 extraneous variables that could affect the response.
%In addition, give a strategy the research should try to deal with each
%one of the given extraneous variables.
%
%\ans{There are many possible answers. Here are some of mine:
%\ben
%\item Total calories in diet, make a controlled variable by setting constant for all rats.
%\item Total amount of fat or carbohydrates in diet, make controlled by setting constant for all rats.
%\item Sex of the rats, either make controlled by using only one sex or using an equal number
%of each sex in each group, or use as a blocking variable.
%\item Litter of the rats or the parents of the rats, randomize the parentage among the treatment groups
%(i.e., don't use separate litters for the different groups).
%\item Exercise given to the rats, attempt to make controlled by setting activity constant.
%\item Living environment, such as the space available, cage materials---these should all be the same
%for a good experiment so would be controlled.
%\item Water intake for rats, make controlled by allowing all rats the same access to water.
%\item Breed of rat, make controlled by using a single breed or species, or else use as a blocking variable.
%\item Vitamin intake, make controlled by using same amount for all rats
%\item Temperature of environment, make controlled
%\item Type of food given, make controlled
%\een
%}
%\een

%\item \textbf{Mathematics curricula.}
%
%Three high schools in Iowa participated in a study to evaluate the 
%effectiveness of a new computer-based mathematics curriculum. In 
%each school, four 24-student sections of freshman algebra were 
%available for study. The two types of instruction (standard 
%curriculum, computer-based curriculum) were randomly assigned to 
%the four sections in each of the three schools, resulting in each 
%instruction method being assigned to two sections within each
%school. 
%%\blue{Cory: 2 of each treatment per school,correct? Perhaps make
%%this point clear?}
%At the end of the 
%term, a standard mathematics achievement test was given to each of 
%the 24 students in each section.
%
%\ben
%\item Is this an experiment or an observational study? Explain 
%briefly.
%\item Identify the experimental units.
%\item Identify all experimental variables, settings of the 
%experimental variables, and treatments.
%\item What type of experimental design is being implemented here?
%\item The twelve sections of freshman algebra constitutes the sample. 
%What is the population?
%
%\een

\een


\een
\end{document}
