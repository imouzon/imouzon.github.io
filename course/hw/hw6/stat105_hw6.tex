% For LaTeX-Box: root = stat105_hw2.tex 
%%%%%%%%%%%%%%%%%%%%%%%%%%%%%%%%%%%%%%%%%%%%%%%%%%%%%%%%%%%%%%%%%%%%%%%%%%%%%%%%
%  File Name: stat105_hw2.tex
%  Purpose:
%
%  Creation Date: 03-09-2015
%  Last Modified: Wed Mar 23 01:03:07 2016
%  Created By:
%%%%%%%%%%%%%%%%%%%%%%%%%%%%%%%%%%%%%%%%%%%%%%%%%%%%%%%%%%%%%%%%%%%%%%%%%%%%%%%%

\documentclass[11pt]{article}
\usepackage{graphicx, fancyhdr}
\usepackage{amsmath, amsfonts}
\usepackage{color}

\newcommand{\blue}[1]{{\color{blue} #1}}

\setlength{\topmargin}{-.5 in} 
\setlength{\textheight}{9 in}
\setlength{\textwidth}{6.5 in} 
\setlength{\evensidemargin}{0 in}
\setlength{\oddsidemargin}{0 in} 
\setlength{\parindent}{0 in}
\newcommand{\ben}{\begin{enumerate}}
\newcommand{\een}{\end{enumerate}}


\lhead{STAT 105, Section A} 
\chead{Homework Assignment 6} 
\rhead{Due Tuesday, March 29} 
\lfoot{Spring 2016} 
\cfoot{\thepage} 
\rfoot{} 
\renewcommand{\headrulewidth}{0.4pt} 
\renewcommand{\footrulewidth}{0.4pt} 
\newcommand{\ans}[1]{{\color{blue} \textbf{Answer: } #1}}
\renewcommand{\ans}[1]{}

\def\Exp#1#2{\ensuremath{#1\times 10^{#2}}}
\def\Case#1#2#3#4{\left\{ \begin{tabular}{cc} #1 & #2 \phantom
{\Big|} \\ #3 & #4 \phantom{\Big|} \end{tabular} \right.}

\begin{document}
\pagestyle{fancy} 

Show \textbf{all} of your work on this assignment and answer each question fully in the given context. \\

\emph{Please} staple your assignment! \\

\ben

\item \textbf{Chapter 5, Section 1, Exercise 6 (page 244)}

\item \textbf{Chapter 5, Section 1, Exercise 8 (page 244)}

\item \textbf{Chapter 5, Section 2, Exercise 1 (page 263, parts (a), (b), (c), and (e) only)}

\item Consider a continuously distributed random variable, $W$, with a probability density function given by
   $$
   f(w) = \begin{cases}
      \frac{1}{5(1 - e^{-2})} e^{-w/5} & 0 \le w \le 10 \\

      0 & \text{otherwise}
   \end{cases}
   $$
   \begin{enumerate}

      \item Graph the probability density function (carefully labeling important features).

      \item Show that the function $f(w)$ is a valid probability density function (i.e., show that (i) $f(w)$ is non-negative and (ii) $\int_{-\infty}^{\infty} f(w) dw = 1$).
         
      \item Find $P(W \le 2)$
         
      \item Find $P(2 \le W \le 5)$

      \item Find $P(5 \le W \le 10)$

      \item Find $P(2 \le W \le 10)$
      
   \end{enumerate}


 \item (\textit{This problem is worth 5 bonus points})
    We know that the probability density function of what is known as a normal random variable with $\mathbb{E}(X) = \mu$ and variance $\sigma^2$ can be written as
    $$
    f(x) = \frac{1}{\sqrt{2 \pi \sigma^2}} e^{-\frac{1}{2 \sigma^2 } (x - \mu)^2}, -\infty < x < \infty
    $$
    and we also know that, as with any random variable, the integral from $-\infty$ to $\infty$ of the function $f(x)$ above must be 1:
    $$
    \int_{-\infty}^{\infty} f(x) dx  = \int_{-\infty}^{\infty} \frac{1}{\sqrt{2 \pi \sigma^2}} e^{-\frac{1}{2 \sigma^2 } (x - \mu)^2} dx  = 1
    $$
    for any value we are given for $\mu$ or $\sigma^2$.

    Using the last fact, to find the value of $ \int_{-\infty}^{\infty} e^{3x -x^2} dx $
\een
\end{document}
