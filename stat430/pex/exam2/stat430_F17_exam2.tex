% For LaTeX-Box: root = stat105_F15_exam1B.tex 
%%%%%%%%%%%%%%%%%%%%%%%%%%%%%%%%%%%%%%%%%%%%%%%%%%%%%%%%%%%%%%%%%%%%%%%%%%%%%%%%
%  File Name: stat105_F15_exam1B.tex
%  Purpose:
%
%  Creation Date: 24-09-2015
%  Last Modified: Mon Oct  2 13:57:00 2017
%  Created By:
%%%%%%%%%%%%%%%%%%%%%%%%%%%%%%%%%%%%%%%%%%%%%%%%%%%%%%%%%%%%%%%%%%%%%%%%%%%%%%%%
\documentclass[addpoints]{examsetup}\usepackage[]{graphicx}\usepackage[]{color}
%% maxwidth is the original width if it is less than linewidth
%% otherwise use linewidth (to make sure the graphics do not exceed the margin)
\makeatletter
\def\maxwidth{ %
  \ifdim\Gin@nat@width>\linewidth
    \linewidth
  \else
    \Gin@nat@width
  \fi
}
\makeatother

\definecolor{fgcolor}{rgb}{0.345, 0.345, 0.345}
\newcommand{\hlnum}[1]{\textcolor[rgb]{0.686,0.059,0.569}{#1}}%
\newcommand{\hlstr}[1]{\textcolor[rgb]{0.192,0.494,0.8}{#1}}%
\newcommand{\hlcom}[1]{\textcolor[rgb]{0.678,0.584,0.686}{\textit{#1}}}%
\newcommand{\hlopt}[1]{\textcolor[rgb]{0,0,0}{#1}}%
\newcommand{\hlstd}[1]{\textcolor[rgb]{0.345,0.345,0.345}{#1}}%
\newcommand{\hlkwa}[1]{\textcolor[rgb]{0.161,0.373,0.58}{\textbf{#1}}}%
\newcommand{\hlkwb}[1]{\textcolor[rgb]{0.69,0.353,0.396}{#1}}%
\newcommand{\hlkwc}[1]{\textcolor[rgb]{0.333,0.667,0.333}{#1}}%
\newcommand{\hlkwd}[1]{\textcolor[rgb]{0.737,0.353,0.396}{\textbf{#1}}}%

\usepackage{framed}
\makeatletter
\newenvironment{kframe}{%
 \def\at@end@of@kframe{}%
 \ifinner\ifhmode%
  \def\at@end@of@kframe{\end{minipage}}%
  \begin{minipage}{\columnwidth}%
 \fi\fi%
 \def\FrameCommand##1{\hskip\@totalleftmargin \hskip-\fboxsep
 \colorbox{shadecolor}{##1}\hskip-\fboxsep
     % There is no \\@totalrightmargin, so:
     \hskip-\linewidth \hskip-\@totalleftmargin \hskip\columnwidth}%
 \MakeFramed {\advance\hsize-\width
   \@totalleftmargin\z@ \linewidth\hsize
   \@setminipage}}%
 {\par\unskip\endMakeFramed%
 \at@end@of@kframe}
\makeatother

\definecolor{shadecolor}{rgb}{.97, .97, .97}
\definecolor{messagecolor}{rgb}{0, 0, 0}
\definecolor{warningcolor}{rgb}{1, 0, 1}
\definecolor{errorcolor}{rgb}{1, 0, 0}
\newenvironment{knitrout}{}{} % an empty environment to be redefined in TeX

\usepackage{alltt}

\usepackage{etoolbox}
\usepackage{tikz,pgfplots}
\usepackage{graphicx, fancyhdr}
\usepackage{amsmath, amsfonts}
\usepackage{color}

%% For LaTeX-Box: root = stat105_exam1_info.tex 
%%%%%%%%%%%%%%%%%%%%%%%%%%%%%%%%%%%%%%%%%%%%%%%%%%%%%%%%%%%%%%%%%%%%%%%%%%%%%%%%
%  File Name: stat105_exam1_info.tex
%  Purpose:
%
%  Creation Date: 24-09-2015
%  Last Modified: Thu Sep 24 13:51:36 2015
%  Created By:
%%%%%%%%%%%%%%%%%%%%%%%%%%%%%%%%%%%%%%%%%%%%%%%%%%%%%%%%%%%%%%%%%%%%%%%%%%%%%%%%
\newcommand{\course}[1]{\ifstrempty{#1}{STAT 105}{STAT 105, Section #1}}
\newcommand{\sectionNumber}{B}
\newcommand{\examDate}{October 1, 2015}
\newcommand{\semester}{FALL 2015}
\newcommand{\examNumber}{II}

\newcommand{\examTitle}{Exam \examNumber}

\runningheader{\course{\sectionNumber}}{Exam \examNumber}{\examDate}
\runningfooter{}{}{Page \thepage of \numpages}

\newcommand{\examCoverPage}{
   \begin{coverpages}
   \centering
   {\bfseries\scshape\Huge Exam I \par}
   \vspace{1cm}
   {\bfseries\scshape\LARGE \course{\sectionNumber} \par}
   {\bfseries\scshape\LARGE \semester \par}

   \vspace{2cm}

   \fbox{\fbox{\parbox{5.5in}{\centering 

      \vspace{.25cm} 
      
      {\bfseries\Large Instructions} \\

      \vspace{.5cm} 

      \begin{itemize}
         \item  The exam is scheduled for 80 minutes, from 8:00 to 9:20 AM. At 9:20 AM the exam will end.\\
         \item  A forumula sheet is attached to the end of the exam. Feel free to tear it off.\\
         \item  You may use a calculator during this exam.\\
         \item  Answer the questions in the space provided. If you run out of room, continue on the back of the page. \\
         \item  If you have any questions about, or need clarification on the meaning of an item on this exam, please ask your instructor. No other form of external help is permitted attempting to receive help or provide help to others will be considered cheating.\\
         \item  {\bfseries Do not cheat on this exam.} Academic integrity demands an honest and fair testing environment. Cheating will not be tolerated and will result in an immediate score of 0 on the exam and an incident report will be submitted to the dean's office.\\
      \end{itemize}

   }}}

   \vspace{2cm}

   \makebox[0.6\textwidth]{Name:\enspace\hrulefill}

   \vspace{1cm}

   \makebox[0.6\textwidth]{Student ID:\enspace\hrulefill}
   \end{coverpages}

}


\newcommand{\course}[1]{\ifstrempty{#1}{STAT 430}{STAT 430}}
\newcommand{\sectionNumber}{B}
\newcommand{\examDate}{November 7, 2015}
\newcommand{\semester}{FALL 2017}
\newcommand{\examNumber}{II}

%%%%%%%%%%%%%%%%%%%%%%%%%%%%%%%%%%%%%%%%%%%%%%%%%%%%%%%%%%%%%%%%%%%%%%%%%%%%%%%%
\IfFileExists{upquote.sty}{\usepackage{upquote}}{}
\begin{document}

%-- : R code (Code in Document)



\examCoverPage

\begin{questions}

\question[5] 
Suppose that $U$ is a uniform random variable on the interval $[0,1]$. Also suppose that $F$ is any invertable cumulative density function. Find the cdf of $X = F^{-1}(U)$.

\question[5] 
Let $X$ be any random variable with mean $\mu$ and variance $\sigma^2$. What is the mean and variance of the random variable $Z = \frac{X - \mu}{\sigma}$?

\question[5] 
For any random variables $X$ and $Y$ show that $Var(X - Y) = Var(X) + Var(Y) - 2 Cov(X, Y)$.

\question[5] 
For any random variables $X$ and $Y$, find that $Var(X \cdot Y)$ in terms of $E(X), E(Y), Var(X), Var(Y)$, and $Cov(X,Y)$.

\question
Suppose that $X_1, X_2, ..., X_n$ are independent random variables with the same expected value $\mu$ and the same variance $\sigma^2$.
\begin{parts}
   \part[5] What is the expected value and variance of $\bar{X} = \frac{1}{n} (X_1 + X_2 + ... + X_n$)?
   \part[5] What is the expected value and variance of $T = X_1 + 2 X_2 + ... + n X_n$?
\end{parts}

\question
Let $X_1$ and $X_2$ are independent, identically distributed random variables with the same probability density function
\[
   f_{X}(x) = \dfrac{\lambda^{\alpha}}{\Gamma(\alpha)} x^{\alpha - 1} e^{-\lambda x}
\]
(i.e., $X_1$ and $X_2$ are gamma random variables).
\begin{parts}
   \part[5] Find the probability density function for $Y_1 = 1/X_1$ (be sure to include the range).
   \part[5] Find the probability density function for $Y_2 = X_1 + X_2$ (be sure to include the range).
   \part[5] What is the expected value of $Z = Y_1 \cdot Y_2$?
\end{parts}

\question
\textbf{Stick Breaking} 
Consider the following scenario: I hold a wooden stick of length $L$ in my left hand and randomly select a point at which to break the stick. I name the portion still in my left hand Stick A and place it on a table. I then pick up what was left of the stick and, holding it in my left hand, break it again by selecting another point at random. I name the piece in my left hand Stick B and place it on the table with Stick A.

\begin{parts}
   \part[5] What is the probability that Stick A is longer than Stick B?
   \part[5] What is the probability that Stick B is longer than the length of the stick left over?
   \part[5] What is the probability that the sum of lengths of Stick A and Stick B is less than $\frac{1}{2} L$?
   \part[5] Which length has the highest variance? The length of Stick A or the length of Stick B?
\end{parts}

\question
Let $X$ be a $N(\mu, \sigma^2)$. 
\begin{parts}
   \part[5] Derive the moment generating function for $X$.
   \part[5] Using the moment generating function for $X$, find the moment generating function for $Y = X - \mu$.
   \part[5] Show that $E(Y^k) = E\left((X - \mu)^k\right)$ is 0 if $k$ is odd.
   \part[5] Find an expression for $E(Y^k) = E\left((X - \mu)^k\right)$ if $k$ is even in terms of $\sigam, \mu$, and $k$.
\end{parts}

\pagebreak
\question[30] \textbf{Gibbs Sampling}
One of the sampling techniques commonly used in Bayesian Analysis is the Gibbs sampler. Essentially, the gibbs sampler is a method of generating multiple observations $x_1, x_2, \ldots, x_n$ from random variables $X_1, X_2, \ldots, X_n$ with joint distribution $f(x_1, x_2, \ldots, x_n)$. The alogorithm works like this:

\begin{itemize}
   \item[1.] With the $i$th observation $x_1, x_2, \ldots, x_n$, generate the $i+1$th observation $x_1^{*}, x_2^{*}, \ldots, x_n^{*}$ as follows:
   \begin{itemize}
      \item[i.] Generate $x_1^{*}$ from the conditional distribution $f(x | X_2 = x_2, X_3 = x_3, \ldots, X_n = x_n)$
      \item[ii.] Generate $x_2^{*}$ from the conditional distribution $f(x | X_1 = x_1^{*}, X_3 = x_3, \ldots, X_n = x_n)$
      \item[iii.] Generate $x_3^{*}$ from the conditional distribution $f(x | X_1 = x_1^{*}, X_2 = x_2^{*}, \ldots, X_n = x_n)$
      \item[] \ldots
      \item[n.] Generate $x_n^{*}$ from the conditional distribution $f(x | X_1 = x_1^{*}, X_2 = x_2^{*}, \ldots, X_{n-1} = x_n^{*})$
   \end{itemize}
\end{itemize}
In this way, each iteration generates values $x_1, \ldots, x_n$ following the joint distribution of $X_1, \ldots, X_n$. As an added bonus, if we look only at the values generated for $x_1$, the values will follow the marginal for $X$.

Suppose that $\theta \sim Beta(3,1)$, $X|\theta, Y=y \sim \text{binom}(y, \theta)$, and $Y|\theta \sim \text{binom}(10, \theta)$. 

Write a function in R that generates samples from the joint distribution of $X, Y$ and $\theta$ using a Gibbs Sampling algorithm. Use it to get 10,000 samples distribution after letting the sampler run 10,000 times to avoid any problems from your initial starting point.
Submit your function, your initial starting value, plots of the 10,000 values for each of the variables. Provide summary statistics including the mean and variance of your samples.

   
\end{questions}

\end{document}
