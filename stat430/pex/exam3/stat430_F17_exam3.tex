% For LaTeX-Box: root = stat105_F15_exam1B.tex 
%%%%%%%%%%%%%%%%%%%%%%%%%%%%%%%%%%%%%%%%%%%%%%%%%%%%%%%%%%%%%%%%%%%%%%%%%%%%%%%%
%  File Name: stat105_F15_exam1B.tex
%  Purpose:
%
%  Creation Date: 24-09-2015
%  Last Modified: Mon Oct  2 13:57:00 2017
%  Created By:
%%%%%%%%%%%%%%%%%%%%%%%%%%%%%%%%%%%%%%%%%%%%%%%%%%%%%%%%%%%%%%%%%%%%%%%%%%%%%%%%
\documentclass[addpoints]{examsetup}\usepackage[]{graphicx}\usepackage[]{color}
%% maxwidth is the original width if it is less than linewidth
%% otherwise use linewidth (to make sure the graphics do not exceed the margin)
\makeatletter
\def\maxwidth{ %
  \ifdim\Gin@nat@width>\linewidth
    \linewidth
  \else
    \Gin@nat@width
  \fi
}
\makeatother

\definecolor{fgcolor}{rgb}{0.345, 0.345, 0.345}
\newcommand{\hlnum}[1]{\textcolor[rgb]{0.686,0.059,0.569}{#1}}%
\newcommand{\hlstr}[1]{\textcolor[rgb]{0.192,0.494,0.8}{#1}}%
\newcommand{\hlcom}[1]{\textcolor[rgb]{0.678,0.584,0.686}{\textit{#1}}}%
\newcommand{\hlopt}[1]{\textcolor[rgb]{0,0,0}{#1}}%
\newcommand{\hlstd}[1]{\textcolor[rgb]{0.345,0.345,0.345}{#1}}%
\newcommand{\hlkwa}[1]{\textcolor[rgb]{0.161,0.373,0.58}{\textbf{#1}}}%
\newcommand{\hlkwb}[1]{\textcolor[rgb]{0.69,0.353,0.396}{#1}}%
\newcommand{\hlkwc}[1]{\textcolor[rgb]{0.333,0.667,0.333}{#1}}%
\newcommand{\hlkwd}[1]{\textcolor[rgb]{0.737,0.353,0.396}{\textbf{#1}}}%

\usepackage{framed}
\makeatletter
\newenvironment{kframe}{%
 \def\at@end@of@kframe{}%
 \ifinner\ifhmode%
  \def\at@end@of@kframe{\end{minipage}}%
  \begin{minipage}{\columnwidth}%
 \fi\fi%
 \def\FrameCommand##1{\hskip\@totalleftmargin \hskip-\fboxsep
 \colorbox{shadecolor}{##1}\hskip-\fboxsep
     % There is no \\@totalrightmargin, so:
     \hskip-\linewidth \hskip-\@totalleftmargin \hskip\columnwidth}%
 \MakeFramed {\advance\hsize-\width
   \@totalleftmargin\z@ \linewidth\hsize
   \@setminipage}}%
 {\par\unskip\endMakeFramed%
 \at@end@of@kframe}
\makeatother

\definecolor{shadecolor}{rgb}{.97, .97, .97}
\definecolor{messagecolor}{rgb}{0, 0, 0}
\definecolor{warningcolor}{rgb}{1, 0, 1}
\definecolor{errorcolor}{rgb}{1, 0, 0}
\newenvironment{knitrout}{}{} % an empty environment to be redefined in TeX

\usepackage{alltt}

\usepackage{etoolbox}
\usepackage{tikz,pgfplots}
\usepackage{graphicx, fancyhdr}
\usepackage{amsmath, amsfonts}
\usepackage{color}

%% For LaTeX-Box: root = stat105_exam1_info.tex 
%%%%%%%%%%%%%%%%%%%%%%%%%%%%%%%%%%%%%%%%%%%%%%%%%%%%%%%%%%%%%%%%%%%%%%%%%%%%%%%%
%  File Name: stat105_exam1_info.tex
%  Purpose:
%
%  Creation Date: 24-09-2015
%  Last Modified: Thu Sep 24 13:51:36 2015
%  Created By:
%%%%%%%%%%%%%%%%%%%%%%%%%%%%%%%%%%%%%%%%%%%%%%%%%%%%%%%%%%%%%%%%%%%%%%%%%%%%%%%%
\newcommand{\course}[1]{\ifstrempty{#1}{STAT 105}{STAT 105, Section #1}}
\newcommand{\sectionNumber}{B}
\newcommand{\examDate}{October 1, 2015}
\newcommand{\semester}{FALL 2015}
\newcommand{\examNumber}{II}

\newcommand{\examTitle}{Exam \examNumber}

\runningheader{\course{\sectionNumber}}{Exam \examNumber}{\examDate}
\runningfooter{}{}{Page \thepage of \numpages}

\newcommand{\examCoverPage}{
   \begin{coverpages}
   \centering
   {\bfseries\scshape\Huge Exam I \par}
   \vspace{1cm}
   {\bfseries\scshape\LARGE \course{\sectionNumber} \par}
   {\bfseries\scshape\LARGE \semester \par}

   \vspace{2cm}

   \fbox{\fbox{\parbox{5.5in}{\centering 

      \vspace{.25cm} 
      
      {\bfseries\Large Instructions} \\

      \vspace{.5cm} 

      \begin{itemize}
         \item  The exam is scheduled for 80 minutes, from 8:00 to 9:20 AM. At 9:20 AM the exam will end.\\
         \item  A forumula sheet is attached to the end of the exam. Feel free to tear it off.\\
         \item  You may use a calculator during this exam.\\
         \item  Answer the questions in the space provided. If you run out of room, continue on the back of the page. \\
         \item  If you have any questions about, or need clarification on the meaning of an item on this exam, please ask your instructor. No other form of external help is permitted attempting to receive help or provide help to others will be considered cheating.\\
         \item  {\bfseries Do not cheat on this exam.} Academic integrity demands an honest and fair testing environment. Cheating will not be tolerated and will result in an immediate score of 0 on the exam and an incident report will be submitted to the dean's office.\\
      \end{itemize}

   }}}

   \vspace{2cm}

   \makebox[0.6\textwidth]{Name:\enspace\hrulefill}

   \vspace{1cm}

   \makebox[0.6\textwidth]{Student ID:\enspace\hrulefill}
   \end{coverpages}

}


\newcommand{\course}[1]{\ifstrempty{#1}{STAT 430}{STAT 430}}
\newcommand{\sectionNumber}{}
\newcommand{\examDate}{December 15, 2015}
\newcommand{\semester}{FALL 2017}
\newcommand{\examNumber}{III}

%%%%%%%%%%%%%%%%%%%%%%%%%%%%%%%%%%%%%%%%%%%%%%%%%%%%%%%%%%%%%%%%%%%%%%%%%%%%%%%%
\IfFileExists{upquote.sty}{\usepackage{upquote}}{}
\begin{document}

%-- : R code (Code in Document)



\examCoverPage

\begin{questions}
\question[5] 
Suppose that $X$ is a Poisson random variable with rate $\lambda$. Find $E(1/(X + 1))$.

\question[5] 
Suppose that $X$ is a uniform on the interval $[0, 1]$. With $Y = \sqrt{X}$, find $E(Y)$ and $Var(Y)$.

\question
Suppose that $Z_1, Z_2, \ldots$ are independent standard normal random variables. 
\begin{parts}
   \part[5] Find the moment generating function of $X_n = \sum_{i = 1}^{n} \frac{1}{3^i} Z_i$.
   \part[5] Using the moment generating function, find the mean and variance of $X_n$.
   \part[5] As $n \rightarrow \infty$, what happens to the distribution of $X_n$?
\end{parts}

\question[10] 
In R, use the Monte Carlo method of integration to estimate the value of $\int_{0}^1 sin(2 \pi x) dx$ with $n = 100$, $n = 1000$ and $n = 10000$. Compare the estimated integral to the exact value.

\question[5] 

(Edited Saturday, Decemeber 9 at 10:30)

Let $\{X_i\}$ be a sequence of independent random variables with $E(X_i) = \mu$ and $Var(X_i) = \sigma_i^2$ (i.e., the variances of the random variables are different). Show that if $\sum_{i=1}^n \sigma_i^2/n^2 \rightarrow 0$ then $\bar{X}$ converges to $\mu$ in probability.

\textit{note: when I refer to $\bar{X}$ I could have used a better notation $\bar{X}_n$ - it was intended to refer to $\frac{1}{n}\sum_{i=1}^n X_i$. As the we take new values in the sequence, we would produce a sequence of these "sample averages" - so $\bar{X}_1$, $\bar{X}_2$, \ldots.}

\question

(Edited Saturday, Decemeber 9 at 10:30)

Consider an sample of $n$ independent random variables with density function
\[
   f(y | \theta) = \frac{\theta}{2} e^{- |y|/\theta }, -\infty < x < \infty
\]
\begin{parts}
   \part[5] Find the MME estimator of $\theta$.
   \part[5] Find the MLE estimator of $\theta$.
\end{parts}

\question
   Suppose that $\epsilon_i$ are independent normal random variables with mean 0 and (unknown) variance $\sigma^2$ for $i = 1, 2, ..., n$.
   For known values of $x_i$, consider the statistical model
   \[
      y_i = \beta_0 + \beta_1 x_i + \beta_2 x_i^2 + \epsilon_i
   \]
   \begin{parts}
      \part[10] Find the maximum likelihood estimates 
      $\hat{\beta}_0$
      $\hat{\beta}_1$
      $\hat{\beta}_2$
      and
      $\hat{\sigma^2}$
      in terms of the observable values of $x_i$ and $y_i$.
      
      \part[5] Suppose that an expiriment is performed and the following observations are collected:
      \begin{table}[h]
         \centering
         \begin{tabular}{ccccccccccccc}
            x &    1 &    2 &     3 &     4 &      5 &      6 &      7 &      8 &       9 &      10 &      11 &      12 \\
            y & 5.63 & 3.41 & -0.92 & -8.96 & -20.75 & -38.23 & -60.54 & -85.79 & -118.26 & -147.34 & -182.94 & -226.32 \\
              & 5.88 & 3.03 & -0.16 & -8.98 & -22.54 & -41.10 & -60.98 & -85.47 & -117.15 & -148.61 & -185.78 & -225.89 \\
         \end{tabular}

      \end{table}
      Using this data, provided the fitted version of the model from part a),
      \[
         \hat{y}_i = \hat{\beta}_0 + \hat{\beta}_1 x_i + \hat{\beta}_2 x_i^2
      \]
      \part[5] In R, create a plot with the fitted values vs the observed values (i.e., the fitted values $\hat{y}_i$ on the x-axis). What does this plot indicate about the overall quality of the fitted model?
      \part[5] Using the fitted model from part b), find the values of the residuals 
      \[
         e_i = y_i - \hat{y}_i
      \]
      and provide a histogram of these values. What does the shape of the residuals histogram indicate about the assumptions we make when fitting the model we used in part b)?
   \end{parts}

\question
  
   Suppose that $\epsilon_i$ are independent normal random variables with mean 0 and (unknown) variance $\sigma^2$ for $i = 1, 2, ..., n$.
   For known values of $x_i$, consider two statistical models: \\

   \textbf{Model A}
   \[
      y_i = \beta_0 e^{ \beta_1 x_i } + \epsilon_i
   \]
   \[
      \epsilon_i \sim N(0, \sigma^2) \text{ (independent) }
   \]

   \textbf{Model B}
   \[
      log(y_i) = \alpha_0 + \alpha_1 x_i + \epsilon_i
   \]
   \[
      \epsilon_i \sim N(0, \sigma^2) \text{ (independent) }
   \]

   \begin{parts}
      \part[10] Using maximum likelihood estimators, provided the estimated parameters of Model A in terms of $x_i$ and $y_i$.
      \part[10] Using maximum likelihood estimators, provided the estimated parameters of Model B in terms of $x_i$ and $y_i$.
      \part[5] Are these models equivalent? That is, can we model an exponential relationship between $y_i$ and $x_i$ by modeling a linear relationship between $log(y_i)$ and $x_i$? Explain.
   \end{parts}

\end{questions}

\end{document}
