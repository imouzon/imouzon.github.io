% For LaTeX-Box: root = stat105_F15_exam1B.tex 
%%%%%%%%%%%%%%%%%%%%%%%%%%%%%%%%%%%%%%%%%%%%%%%%%%%%%%%%%%%%%%%%%%%%%%%%%%%%%%%%
%  File Name: stat105_F15_exam1B.tex
%  Purpose:
%
%  Creation Date: 24-09-2015
%  Last Modified: Mon Oct  2 13:58:12 2017
%  Created By:
%%%%%%%%%%%%%%%%%%%%%%%%%%%%%%%%%%%%%%%%%%%%%%%%%%%%%%%%%%%%%%%%%%%%%%%%%%%%%%%%
\documentclass[addpoints]{examsetup}

\usepackage{etoolbox}
\usepackage{tikz,pgfplots}
\usepackage{graphicx, fancyhdr}
\usepackage{amsmath, amsfonts}
\usepackage{color}

%% For LaTeX-Box: root = stat105_exam1_info.tex 
%%%%%%%%%%%%%%%%%%%%%%%%%%%%%%%%%%%%%%%%%%%%%%%%%%%%%%%%%%%%%%%%%%%%%%%%%%%%%%%%
%  File Name: stat105_exam1_info.tex
%  Purpose:
%
%  Creation Date: 24-09-2015
%  Last Modified: Thu Sep 24 13:51:36 2015
%  Created By:
%%%%%%%%%%%%%%%%%%%%%%%%%%%%%%%%%%%%%%%%%%%%%%%%%%%%%%%%%%%%%%%%%%%%%%%%%%%%%%%%
\newcommand{\course}[1]{\ifstrempty{#1}{STAT 105}{STAT 105, Section #1}}
\newcommand{\sectionNumber}{B}
\newcommand{\examDate}{October 1, 2015}
\newcommand{\semester}{FALL 2015}
\newcommand{\examNumber}{II}

\newcommand{\examTitle}{Exam \examNumber}

\runningheader{\course{\sectionNumber}}{Exam \examNumber}{\examDate}
\runningfooter{}{}{Page \thepage of \numpages}

\newcommand{\examCoverPage}{
   \begin{coverpages}
   \centering
   {\bfseries\scshape\Huge Exam I \par}
   \vspace{1cm}
   {\bfseries\scshape\LARGE \course{\sectionNumber} \par}
   {\bfseries\scshape\LARGE \semester \par}

   \vspace{2cm}

   \fbox{\fbox{\parbox{5.5in}{\centering 

      \vspace{.25cm} 
      
      {\bfseries\Large Instructions} \\

      \vspace{.5cm} 

      \begin{itemize}
         \item  The exam is scheduled for 80 minutes, from 8:00 to 9:20 AM. At 9:20 AM the exam will end.\\
         \item  A forumula sheet is attached to the end of the exam. Feel free to tear it off.\\
         \item  You may use a calculator during this exam.\\
         \item  Answer the questions in the space provided. If you run out of room, continue on the back of the page. \\
         \item  If you have any questions about, or need clarification on the meaning of an item on this exam, please ask your instructor. No other form of external help is permitted attempting to receive help or provide help to others will be considered cheating.\\
         \item  {\bfseries Do not cheat on this exam.} Academic integrity demands an honest and fair testing environment. Cheating will not be tolerated and will result in an immediate score of 0 on the exam and an incident report will be submitted to the dean's office.\\
      \end{itemize}

   }}}

   \vspace{2cm}

   \makebox[0.6\textwidth]{Name:\enspace\hrulefill}

   \vspace{1cm}

   \makebox[0.6\textwidth]{Student ID:\enspace\hrulefill}
   \end{coverpages}

}


\newcommand{\course}[1]{\ifstrempty{#1}{STAT 430}{STAT 430}}
\newcommand{\sectionNumber}{}
\newcommand{\examDate}{October 3, 2017}
\newcommand{\semester}{FALL 2017}
\newcommand{\examNumber}{I}

%%%%%%%%%%%%%%%%%%%%%%%%%%%%%%%%%%%%%%%%%%%%%%%%%%%%%%%%%%%%%%%%%%%%%%%%%%%%%%%%

\begin{document}

\examCoverPage

\begin{questions}


\question \textbf{Sets Proof} \\ For the following question, assume that $S$ is the universe, and $A$, $B$, and $C$ are subsets of $S$.

   \begin{parts}
      \part[4] Prove or disprove that $ A \cap B \subseteq A \subset A \cup B $
      \vspace{5cm}
      \part[4] Prove or disprove that if $A \not\subseteq B$ and $B \not\subseteq C$ then $A \notsubset C$.
      \vspace{5cm}
      \part[4] Prove or disprove that $(A \cap B^{c}) \cup (B \cap C^{c}) = A \cap C^{c}$
      \vspace{5cm}
   \end{parts}

\newpage

\question[8]
   Suppose that we can call a collection of sets, $\mathcal{F}$ a "sigma algebra" if 
   \begin{enumerate}
      \item $\emptyset \in \mathcal{F}$
      \item if $A \in \mathcal{F}$ then $A^{C} \in \mathcal{F}$
      \item if $A_1, A_2, \cdots \in \mathcal{F}$ then $A_1 \cup A_2 \cup \ldots \in \mathcal{F}$.
   \end{enumerate}
   Suppose that $A \cap B = \emptyset$. Write the smallest possible sigma-algebra containing $A$ and $B$.

\newpage
%-- : R code (Code in Document)
\question
Consider the following scenario: Suppose that a fair coin is tossed. If the toss results in "Heads" facing up, then a fair 6-sided die is rolled. If the toss results in a "Tails" facing up, a fair 4-sided die is rolled.

\begin{parts}
\part[2] Write the sample space of possible outcomes for this experiment, using "H\" and "T\" to denote heads and tails and integers to denote the roll of the die.
\vspace{5cm}
\part
Determine the following probabilities:
\vspace{1cm}
  \begin{subparts}
      \subpart[2] Find $P\left(\text{"the coin toss results in heads"}\right)$.
      \vspace{2cm}
      \subpart[2] Find $P\left(\text{"an even number is rolled"}\right)$.
      \vspace{2cm}
      \subpart[2] Find $P\left(\text{"an even number is rolled"|"the roll was less than 3"}\right)$.
      \vspace{2cm}
      \subpart[2] Find $P\left(\text{"The coin toss resulted in heads" | "the die roll was less than 3"}\right)$
      \vspace{2cm}
  \end{subparts}

\end{parts}

\newpage

\question

   \begin{parts}
      \part[4] Suppose $X \sim \text{Binom}(4,.2)$. Find $P(X = 3)$.
      \vspace{4cm}
      \part[4] Suppose $Y \sim \text{Geometric}(.9)$. Find $P(Y \ge 10)$.
      \vspace{4cm}
      \part[4] Suppose $M \sim \text{Geometric}(.2)$. Find $P(M \ge 3|M \le 5)$.
      \vspace{4cm}
      \part[4] Using what you know about probability mass functions, evaluate the sum \[ \sum_{k=0}^{\infty} \left(\frac{2}{3}\right)^k \]
      \vspace{4cm}
      \part[4] Suppose that $T$ is an exponential random variable with density function $f(t) = 1.2 e^{-1.2 t}$ with $ t \ge 0$. Sketch the cumulative distribution for $T$. Use it to illustrate the probability of $T$ being less than 4.
      \vspace{4cm}
      \part[4] Suppose that $T$ is an exponential as in the previous part. If we know that $T$ is greater than $5$, what is the probability that $T$ is less than 10?
      \vspace{4cm}
      \part[4] Suppose $R \sim \text{Beta}(3,4)$. Find $P(R \ge .5 | R \ge .2)$.
      \vspace{4cm}
      \part[4] Using what you know about probability density functions, evaluate the integral \[ \int_{-\infty}^{\infty} e^{-2 x^2} dx \]
      \vspace{4cm}
   \end{parts}

\newpage
\question 

Suppose that $X$ is a continuous random variable with probability density function (pdf):
$$
f(x) = 
\begin{cases}
   3 e^{-6x} & x < 0 \\
   3 e^{-6x} & x \ge 0
\end{cases}
$$

We refer to such a random variable as a mirrored exponential random variable.

\begin{parts}
   \part Provide a sketch of the probability density function, $f(x)$.
   \vspace{4cm}
   \part Show that $f(x)$ is a valid pdf.
   \vspace{2cm}
   \part What is the probability that $X$ takes a value greater than 3?
   \vspace{2cm}
   \part What is the probability that $X$ takes a value between -3 and 3?
   \vspace{2cm}
   \part Derive the cumulative density function, $F(x)$ (consider writing it as a piecewise function).
\end{parts}

\end{questions}

\end{document}
